
\documentclass{article}


\usepackage{listings}

\title{Test, test me do: unit testing for infectious disease modelling}
\author{Tim CD Lucas, Tim Pollington, Deirdre Hollingsworth, et al.}


\begin{document}



% use covid Ferguson as mitigating example.
% https://twitter.com/neil_ferguson/status/1241835454707699713?s=19

% frameworks
%.  testthat and packages
%.  test in Julia
%.  add python

% continuous integration

% good practice
%.   small functions

% what to test
%.  edge cases. 0, others?
%.  types
%.  dims
%.  bounds


% how to test random
%.   non random bit. p= 0
%.   separate the random bit
%.   large numbers and bounds

\maketitle


\section{ example}

Consider a multistrain system, where a population of $N$ individuals who get reinfected once per time step. 
Each individual is defined by the strain they are currently infected with $I_{it} \in \{'a', 'b', 'c'\}$ and so the the population is defined by the length $N$ vector of states $\mathbf{I_t}$.
Then each time step, the infection status of each individual is updated
$$I_{it} = Unif(\mathbf{I{t-1}}).$$

Here then is our first attempt at implementing this model.

\begin{lstlisting}

N <- 99
T <- 200
I <- matrix(NA, ncom = N, nrow = T)
I[1, ] <- rep(c("a", "b", "c"), N / 3)

for(t in seq(2, T)){
  I[t, ] <- sample(I[t - 1, ], N)
}
\end{lstlisting}


\end{document}




